\documentclass[11pt]{article}
\usepackage[T1]{fontenc}
\usepackage[utf8]{inputenc}
\usepackage[english]{babel}
\usepackage[dvipsnames]{xcolor}
\usepackage[a4paper]{geometry}
\usepackage{microtype}
\usepackage{lastpage}
\usepackage{listings}
\lstset{	basicstyle=\small\sffamily,
		language=R,
		numbers=left,
		numberstyle=\tiny\bfseries\sffamily,
		stepnumber=1,
		tabsize=4,
		}
\usepackage{fancyhdr}
\pagestyle{fancy}
\renewcommand{\headrulewidth}{0pt}
\renewcommand{\footrulewidth}{0pt}
\fancyhf{}
\fancyfoot[C]{\bfseries\thepage\ of \pageref{LastPage}}
\fancypagestyle{plain}
{\fancyhf{}
\fancyfoot[C]{\bfseries\thepage\ of \pageref{LastPage}}}
\usepackage{booktabs}
\usepackage{hyperref}
\usepackage{bookmark}
\begin{document}
\title{Codebook}
\author{\textsf{SabaDeMa}}
\date{}
\maketitle
\section{Introduction}
This file explains the code of the script for the assignment. It is organized by steps, labelled with \textsf{``Step''} and written in the script as comments. Before you start of course you have to unzip the folder with the data in the working directory in a a folder called \textsf{"UCI HAR DATASET"}.

If not the case, and you prefer to unzip directly in the working directory, you have to delete from \textsf{read.table} commands the \textsf{"UCI HAR DATASET/"} in order to make the code work well.

I preferred to keep variables of every step, this is just a personal choice but if users might want to delete objects or overwrite them I think that my code can be modified very easily.

\section{Details of the steps}
The first command load \textsf{dplyr} package and (I think) no explanations are needed.

\begin{description}
\item[\textsf{Step1:}] read all files in \textsf{\bfseries R} by using the read.table function.
\item[\textsf{Step2:}] convert all objects to \textsf{tbl\_df} objects in order to use them in an easier way and allow functions like \textsf{group\_by} to work properly.
\item[\textsf{Step3:}] select only the second column (the name, which is what we need) of  feature to use the result as names for the columns of \textsf{dataTrain} and \textsf{dataTest}, but before of that it must be converted into a vector and transposed with the \textsf{t()} function.
Create names for the subjects and activities columns respectively \textsf{``Sub''} and \textsf{``Act''}.
\item[\textsf{Step4:}] create a data frame for test and train with the data, the sub and the activities columns. Merge data frames together (\textsf{data\_train\_fin} and \textsf{data\_test\_fin}) to have a total data frame with all the sets names data\_total.
\item[\textsf{Step5:}] select with the \textsf{grepl} function all columns that have the words mean and std inside their names.
\item[\textsf{Step6:}] in order to have a more readable and shorter columns names this step does some abbreviations and replacements. It delete \textsf{tBody}, \textsf{fBody} and parenthesis (by escaping them with a double \textbackslash) from columns names and gives \textsf{``GY''} instead of \textsf{``Gyro''} and \textsf{``AC''} instead of \textsf{``Acc''}.
\item[\textsf{Step7:}] this dataset has some issues due to some identical columns names, this can be verified with the following commands.
These commands show that the problems are not due to modifications in the script  but they were there when the original dataset was created.
\begin{lstlisting}
> similarcol <- anyDuplicated(data_total, MARGIN = 2)
> similarcol
[1] 0
\end{lstlisting}
Columns have same names but with the previous commands we know that they are not the same things so I think that is better to keep them but in order to avoid problems, change name to the duplicates. That is done with this command:
\begin{lstlisting}
por <- make.names(nomi, unique = TRUE)
\end{lstlisting}
\item[\textsf{Step8:}] change levels from numbers 1 to 6 to activities names (\textsf{Walking}, \textsf{Walking\_Up}, etc.)
\item[\textsf{Step9:}] group the dataset by using the \textsf{group\_by} function.
\item[\textsf{Step10:}] create a dataset named \textsf{data\_4} with the mean for each variable grouped for each subject and activity.
\item[\textsf{Step11:}] write \textsf{data\_4} dataset to an external file located in the woking directory named \textsf{data.txt}.
\end{description}

\section{Output file}
The output file is \textsf{data\_4}. It has 14580 observations in total, 180 rows and 81 columns.
Values of first column named \textsf{Act}, which describes the activities done by subjects are: \textsf{Walking}, \textsf{Walking\_Up},  \textsf{Walking\_down}, \textsf{Sitting}, \textsf{Standing}, \textsf{Lying}.

Values of the second column named \textsf{Sub}, which represents the subjects are from 1 to 30, each number is a subjects.

Names of all columns are:
\begin{lstlisting}
> names(data_4)
 [1] "Act"                   "Sub"                  
 [3] "ACmeanX"               "ACmeanY"              
 [5] "ACmeanZ"               "ACstdX"               
 [7] "ACstdY"                "ACstdZ"               
 [9] "tGravityACmeanX"       "tGravityACmeanY"      
[11] "tGravityACmeanZ"       "tGravityACstdX"       
[13] "tGravityACstdY"        "tGravityACstdZ"       
[15] "ACJerkmeanX"           "ACJerkmeanY"          
[17] "ACJerkmeanZ"           "ACJerkstdX"           
[19] "ACJerkstdY"            "ACJerkstdZ"           
[21] "GYmeanX"               "GYmeanY"              
[23] "GYmeanZ"               "GYstdX"               
[25] "GYstdY"                "GYstdZ"               
[27] "GYJerkmeanX"           "GYJerkmeanY"          
[29] "GYJerkmeanZ"           "GYJerkstdX"           
[31] "GYJerkstdY"            "GYJerkstdZ"           
[33] "ACMagmean"             "ACMagstd"             
[35] "tGravityACMagmean"     "tGravityACMagstd"     
[37] "ACJerkMagmean"         "ACJerkMagstd"         
[39] "GYMagmean"             "GYMagstd"             
[41] "GYJerkMagmean"         "GYJerkMagstd"         
[43] "ACmeanX.1"             "ACmeanY.1"            
[45] "ACmeanZ.1"             "ACstdX.1"             
[47] "ACstdY.1"              "ACstdZ.1"             
[49] "ACmeanFreqX"           "ACmeanFreqY"          
[51] "ACmeanFreqZ"           "ACJerkmeanX.1"        
[53] "ACJerkmeanY.1"         "ACJerkmeanZ.1"        
[55] "ACJerkstdX.1"          "ACJerkstdY.1"         
[57] "ACJerkstdZ.1"          "ACJerkmeanFreqX"      
[59] "ACJerkmeanFreqY"       "ACJerkmeanFreqZ"      
[61] "GYmeanX.1"             "GYmeanY.1"            
[63] "GYmeanZ.1"             "GYstdX.1"             
[65] "GYstdY.1"              "GYstdZ.1"             
[67] "GYmeanFreqX"           "GYmeanFreqY"          
[69] "GYmeanFreqZ"           "ACMagmean.1"          
[71] "ACMagstd.1"            "ACMagmeanFreq"        
[73] "BodyACJerkMagmean"     "BodyACJerkMagstd"     
[75] "BodyACJerkMagmeanFreq" "BodyGYMagmean"        
[77] "BodyGYMagstd"          "BodyGYMagmeanFreq"    
[79] "BodyGYJerkMagmean"     "BodyGYJerkMagstd"     
[81] "BodyGYJerkMagmeanFreq"
\end{lstlisting}
\end{document}